\documentclass{article}

\usepackage[a4paper]{geometry}

\usepackage{amssymb}
\usepackage{amsmath}
\usepackage{amsthm}
\usepackage{amsfonts}

\usepackage{multicol}

% see https://tex.stackexchange.com/a/39755
% Simulated package file
\begin{filecontents}{envcode.sty}
      \newcommand\NewEnvCode[2]{%
       \expandafter\def\csname code@#1\endcsname{#2}%
       \expandafter\def\csname change@code@#1\endcsname{%
        \expandafter\let\expandafter\Code\csname code@#1\endcsname
       }%
      }
      
      \newcommand\MakeDefault{%
       \expandafter\let\expandafter\code@@default\csname code@\@currenvir\endcsname
      }
      
      \newcommand\RunEnvCode{%
       \let\Code=\code@@default
       \csname change@code@\@currenvir\endcsname
       \Code
      }
      
      \AtBeginDocument{\MakeDefault}
\end{filecontents}
\usepackage{envcode}

\newtheorem{definition}{Definition}[section]
\newtheorem{lemma}{Lemma}[section]
\newtheorem{theorem}{Theorem}[section]
\newtheorem{corollary}{Corollary}[section]
\newtheorem{proposition}{Proposition}[section]
\newtheorem{question}{Question}[section]

\NewEnvCode{document}{default code}
\NewEnvCode{theorem}{ theorem }
\NewEnvCode{lemma}{ lemma }
\NewEnvCode{corollary}{ corollary }
\NewEnvCode{proposition}{ proposition }
\NewEnvCode{definition}{ definition }
\NewEnvCode{question}{ question }

\title{Addition of Algebraic Element}
\author{Zifan Hua}

\begin{document}
    \begin{question}
        Proof that $ \sqrt{1} + \sqrt{2} + \sqrt{3} \dots + \sqrt{n} $ for $ n > 2 $ is irrational.
    \end{question}

    However, if we consider the linear combination of 
    $ \sqrt{1} $, $ \sqrt{2} $, $ \sqrt{3} $ $ \dots $ $ \sqrt{n} $ over $ \mathbb{Q} $, 
    then it is likely that we will get a similar result as we  kick out some special cases.

    \begin{question}
        Given $ n $ positive integers, $ a_1 < a_2 < \dots < a_n $, 
        such that $ \sqrt{\frac{a_{j}}{a_{i}}} \notin \mathbb{Q} $ for all $ i < j $ 
        and $ \sqrt{a_{i}} \notin \mathbb{Q} $ for all $i$. 
        Also given $ \lambda_{1}, \lambda_{2} \dots \lambda_{n} \in \mathbb{Q} $. 
        Prove that $ \sum_{i=1}^{n} \lambda_{i} \sqrt{a_{i}} \in \mathbb{Q}$ 
        if and only if $ \lambda_{1} = \lambda_{2} = \dots = \lambda_{n} = 0$.
    \end{question}

    \begin{proof}
        Suppose this stand for $ n=k $, then it is suffice to prove $ n=k+1 $.

        Suppose there exist $ \lambda_{1}, \lambda_{2} \dots \lambda_{k+1} \in \mathbb{Q} $
        such that
        \begin{equation*}
            \sum_{i=1}^{k+1} \lambda_{i}\sqrt{a_{i}} = q \in \mathbb{Q}
        \end{equation*}

        If $\lambda_{j} = 0 $, then
        $ -\lambda_{j}\sqrt{a_{j}} + \sum_{i=1}^{k+1} \lambda_{i}\sqrt{a_{i}} = q \in \mathbb{Q} $,
        and we are done according to the assumption.

        If $\lambda_{j} \neq 0 $ for all $j$, then we have
        \begin{equation*}
            \sum_{i=1}^{k} \lambda_{i}\sqrt{a_{i}} = q - \lambda_{k+1}\sqrt{a_{k+1}}
        \end{equation*}

        Clearly, $ q - \lambda_{k+1}\sqrt{a_{k+1}} $
        is a root of a irreducible quadratic polynomial $ g \in \mathbb{Q}[x] $.
        And the polynomial must have another root which is $ q + \lambda_{k+1}a_{k+1} $.

        Also, the polynomial
        \begin{equation*}
            f = \prod (x - ( \pm \lambda_{1}\sqrt{a_{1}} + \pm \lambda_{2}\sqrt{a_{2}} + \dots + \pm \lambda_{n}\sqrt{a_{n}} ))
        \end{equation*}
        is also in $ \mathbb{Q}[x] $ by induction.

        Also, 
        \begin{eqnarray*}
            f( q - \lambda_{k+1}\sqrt{a_{k+1}} ) &=& f( \sum_{i=1}^{k+1} \lambda_{i}\sqrt{a_{i}} ) \\
            &=& 0
        \end{eqnarray*}

        As $ g $ is irreducible over $ \mathbb{Q} $,
        $ g $ must divides $f$. Thus
        \begin{equation*}
            f( q + \lambda_{k+1}\sqrt{a_{k+1}} ) = 0
        \end{equation*}

        Thus, there exits $ \alpha_{1}, \alpha_{2} \dots \alpha_{n} $
        which is either $ 1 $ or $ -1 $ such that
        \begin{equation*}
            \sum_{i=1}^{k} \alpha_{i} \lambda_{i}\sqrt{a_{i}} = q + \lambda_{k+1}\sqrt{a_{k+1}}
        \end{equation*}

        Thus
        \begin{equation*}
            \begin{cases}
                \sum_{i=1}^{k} \alpha_{i} \lambda_{i}\sqrt{a_{i}} = q + \lambda_{k+1}\sqrt{a_{k+1}} \\
                \sum_{i=1}^{k} \lambda_{i}\sqrt{a_{i}} = q - \lambda_{k+1}\sqrt{a_{k+1}}
            \end{cases}
        \end{equation*}
        \begin{eqnarray*}
            \sum_{i=1}^{k} \alpha_{i} \lambda_{i}\sqrt{a_{i}} + \sum_{i=1}^{k} \lambda_{i}\sqrt{a_{i}}
            &=& q + \lambda_{k+1}\sqrt{a_{k+1} }+ q - \lambda_{k+1}\sqrt{a_{k+1}} \\
            \sum_{i=1}^{k} ( \alpha_{i} + 1 ) \lambda_{i}\sqrt{a_{i}} &=& 2q \in \mathbb{Q} \\
        \end{eqnarray*}

        By the induction assumption,
        we have $ (\alpha_{i} + 1)\lambda_{i} = 0 $ for all $ i \le k $.
        As $ \lambda_{i} \neq 0 $ for all $ i $,
        we have $ \alpha_{i} = -1 $ for all $ i \le k $.

        Thus,
        \begin{eqnarray*}
            \sum_{i=1}^{k} ( \alpha_{i} + 1 ) \lambda_{i}\sqrt{a_{i} } &=& 0
        \end{eqnarray*}
        which implies $ q = 0 $.

        Thus,
        \begin{eqnarray*}
            \sum_{i=1}^{k+1} \lambda_{i}\sqrt{a_{i}} &=& 0 \\
            \sqrt{a_{k+1}} (\sum_{i=1}^{k+1} \lambda_{i}\sqrt{a_{i}}) &=& 0 \\
            \sum_{i=1}^{k} \lambda_{i}\sqrt{a_{k+1}a_{i}} &=& - \lambda_{k+1} a_{k+1} \in \mathbb{Q}\\
        \end{eqnarray*}

        The only thing left to proof is that
        $ a_{k+1}a_{i}, i \le k $ satisfy the assumption in the question,
        and then the induction assumption will apply,
        which proves that $ \lambda_{i} = 0, i \le k $.

        Given any $ i < j $, we have
        \begin{eqnarray*}
            \sqrt{\frac{a_{j}a_{k+1}}{a_{i}a_{k+1}}} &=& \sqrt{\frac{a_{j}}{a_{i}}} \notin \mathbb{Q} \\
            \sqrt{a_{j}a_{k+1}} &=& a_{k+1} \sqrt{\frac{a_{j}}{a_{k+1}}} \notin \mathbb{Q}
        \end{eqnarray*}
    \end{proof}

    As, we does not use any special property of $ \mathbb{Q} $
    other then the fact that it is a field.
    So, the proof should be valid for any field $ \mathbb{F} $ that characteristic equals 0,
    if we rephrase the question in term of $ \mathbb{F} $.

    \begin{question}
        Given $ n $ distinct element $ a_{1}, a_{2}, \dots, a_{n} \in \mathbb{F} $,
        let $ b_{i} $ be a root of $ x^{2} - a_{i} $ for all $ i $.
        And $ b_{i}b_{j}^{-1} \notin \mathbb{F} $ for all $ i \neq j $,
        and $ b_{i} \notin \mathbb{F} $ for all $ i $.
        Then given $ \lambda_{1}, \lambda_{2} \dots \lambda_{n} \in \mathbb{F} $,
        $ \sum_{i=0}^{n} \lambda_{i}b_{i} \in \mathbb{F} $
        if and only if $ \lambda_{i} = 0 $ for all $ i $.
    \end{question}

    If we try to shift $ b_{k} $ to $ b_{k} + c_{k} $
    such that $ c_{k} \in \mathbb{F} $,
    then the result clearly still stand.
    So, the previous question can be generalised a bit.

    \begin{question}
        Given $ n $ distinct irreducible quadratic polynomial
        $ f_{1}, f_{2} \dots f_{n} $ in $ \mathbb{F}[x] $.

        And given any $ i \neq j $, $ f_{j} $ does not have root in $ \mathbb{F}[x]/f_{i} $.

        And $ b_{i} $ be all roots of $ f_{i} $ for all $i$.
        Then given $ \lambda_{1}, \lambda_{2} \dots \lambda_{n} \in \mathbb{F} $,
        $ \sum_{i=0}^{n} \lambda_{i}b_{i} \in \mathbb{F} $
        if and only if $ \lambda_{i} = 0 $ for all $ i $.
    \end{question}

    If we are not satisfied with the quadratic polynomials,
    we can try to generalise the question to any polynomial.
    However, we may need to have a more strict condition on the roots.

    \begin{question}
        Given $ n $ distinct irreducible polynomial
        $ f_{1}, f_{2} \dots f_{n} $ in $ \mathbb{F}[x] $
        with orders greater or equal than 2.

        Given any $ i \neq j$. $ f_{j} $ does not have root in $ \mathbb{F}[x]/f_{i} $.
        
        And $ b_{i} $ be all roots of $ f_{i} $ for all $i$.
        Then given $ \lambda_{1}, \lambda_{2} \dots \lambda_{n} \in \mathbb{F} $,
        $ \sum_{i=0}^{n} \lambda_{i}b_{i,1} \in \mathbb{F} $
        if and only if $ \lambda_{i} = 0 $ for all $ i $.
    \end{question}

    To prove the above question, we need another definition and some relating lemma.

    \begin{definition}
        Given a polynomial $ f(x) \in \mathbb{F}[x] $,
        \begin{equation*}
            f(x) = x^{n} + a_{1}x^{n-1} + a_{2}x^{n-2} + \dots + a_{n-1}x + a_{n}
        \end{equation*}

        Define the derivative $ f'(x) $ of $ f(x) $ as
        \begin{equation*}
            f'(x) = n x^{n-1} + (n-1)a_{1}x^{n-2} + (n-2)a_{2}x^{n-3} + \dots + 2a_{n-2}x + a_{n-1}
        \end{equation*}
    \end{definition}

    Through some simple calculation, we can prove the following lemma.
    \begin{lemma}
        Given polynomials $ f(x), g(x) \in \mathbb{F}[x] $,
        \begin{equation*}
            (f(x)g(x))' = f'(x)g(x) + f(x)g'(x)
        \end{equation*}
    \end{lemma}

    \begin{lemma}
        Given a irreducible polynomial $ f(x) \in \mathbb{F}[x] $,
        and a field $ \mathbb{K} $,
        such that $ \mathbb{F} \subseteq \mathbb{K} $
        and $ f(x) $ can be factorized into product of linear factors in $ \mathbb{K} $.

        Chose a root $ \alpha $ of $ f(x) $ in $ \mathbb{K} $.
        Then $ (x-\alpha)^2 $ divides $ f(x) $ in $ \mathbb{K} $,
        if and only if $ f'(x) = 0 $.
    \end{lemma}

    As $ f'(x) $ is still a polynomial in $ \mathbb{F}[x] $,
    and we can choose arbitrary $ \mathbb{K} $.
    This lemma implies that irreducible polynomial $ f(x) $ have distinct roots
    if and only if $ f'(x) \neq 0 $.

    \begin{proof}
        If $ (x-\alpha)^2 $ divides $ f(x) $ in $ \mathbb{K} $.

        Let,
        \begin{equation*}
            f(x) = (x-\alpha)^{2}g(x)
        \end{equation*}

        Then,
        \begin{equation*}
            f'(x) = 2(x-\alpha)g(x) + (x-\alpha)^{2}g'(x)
        \end{equation*}

        Then $ f'(\alpha) = 0 $.
        As $ f(x) $ is irreducible,
        and $ f(x) $ and $ f'(x) $ have common root.
        So, $ f(x) $ divides $f'(x)$.

        As, the degree of $ f'(x) $ is less than $ f(x) $,
        then $ f'(x) = 0 $.

        Conversely, if $ f'(x) = 0 $.

        Let,
        \begin{equation*}
            f(x) = (x-\alpha)g(x)
        \end{equation*}

        Then,
        \begin{equation*}
            f'(x) = g(x) + (x-\alpha)g'(x)
        \end{equation*}

        Then,
        \begin{eqnarray*}
            0 & = & f'(\alpha) \\
            & = & g(\alpha) + (\alpha-\alpha)g'(\alpha) \\
            & = & g(\alpha)
        \end{eqnarray*}

        So, $ \alpha $ is a root of $ g(x) $.

        Thus, $ (x-\alpha)^2 $ divides $ f(x) $ in $ \mathbb{K} $.
    \end{proof}

    \begin{lemma}
        Given distinct irreducible polynomials $ f(x), g(x) \in \mathbb{F}[x] $,
        where the characteristic of $ \mathbb{F} $ equals $ 0 $.

        Chose any field $ \mathbb{K} $, such that $ \mathbb{F} \subseteq \mathbb{K} $,
        and $ f(x) $ and $ g(x) $ can be factorized into product of linear factors in $ \mathbb{K} $.

        Let $ a_{1}, a_{2} \dots a_{n} $ be all roots of $ f(x) $ in $ \mathbb{K} $,
        and $ b_{1}, b_{2} \dots b_{m} $ be all roots of $ g(x) $ in $ \mathbb{K} $.

        Then,
        \begin{equation*}
            h(x) = \prod_{i=1}^{n}\prod_{j=1}^{m}(x-(a_{i} + b_{j}))
        \end{equation*}
        have coefficients in $ \mathbb{F} $,
    \end{lemma}
\end{document}