\documentclass{article}

\usepackage[a4paper]{geometry}

\usepackage{amssymb}
\usepackage{amsmath}
\usepackage{amsthm}
\usepackage{amsfonts}

\usepackage{multicol}

% See https://tex.stackexchange.com/a/39755
% Simulated package file
\begin{filecontents}{envcode.sty}
    \newcommand\NewEnvCode[2]{%
        \expandafter\def\csname code@#1\endcsname{#2}%
        \expandafter\def\csname change@code@#1\endcsname{%
            \expandafter\let\expandafter\Code\csname code@#1\endcsname
        }%
    }
    
    \newcommand\MakeDefault{%
        \expandafter\let\expandafter\code@@default\csname code@\@currenvir\endcsname
    }
    
    \newcommand\RunEnvCode{%
        \let\Code=\code@@default
        \csname change@code@\@currenvir\endcsname
        \Code
    }
    
    \AtBeginDocument{\MakeDefault}
\end{filecontents}
\usepackage{envcode}

\newtheorem{definition}{Definition}[section]
\newtheorem{lemma}{Lemma}[section]
\newtheorem{theorem}{Theorem}[section]
\newtheorem{corollary}{Corollary}[section]
\newtheorem{proposition}{Proposition}[section]
\newtheorem{question}{Question}[section]
\newtheorem{example}{Example}[section]

\NewEnvCode{document}{default code}
\NewEnvCode{theorem}{ theorem }
\NewEnvCode{lemma}{ lemma }
\NewEnvCode{corollary}{ corollary }
\NewEnvCode{proposition}{ proposition }
\NewEnvCode{definition}{ definition }
\NewEnvCode{question}{ question }
\NewEnvCode{example}{ example }

\begin{document}
\begin{enumerate}
    \item [3.13]
          \begin{enumerate}
              \item [(i)] If $H$ is a proper subgroup of a finite group $G$,
                    then $G$ is not the union of all the conjugates of $H$.
              \item [(ii)] If $G$ is a finite group with conjugacy classes
                    $C_1$, $\dots$, $C_{m}$ and if $g_{i} \in C_{i}$,
                    then $G = \langle g_{1}, \dots, g_{m} \rangle$
          \end{enumerate}
          \begin{proof}
              \begin{enumerate}
                  \item [(i)] We consider the stabilizer of $H$
                        regarding the conjugate action of $G$ on the whole set of subgroups of $G$.
                        \begin{equation}
                            N(H) = \{g \in G \mid gHg^{-1} = H\}
                        \end{equation}

                        Clearly, as $H \subset N(H)$, $|H| \le |N(H)|$ .

                        And clearly $\left[G: N(H)\right] = \frac{|G|}{|N(H)|}$ is the number of conjugates of $H$ in $G$.

                        Consider the number of elements in the union of all the conjugates of $H$.
                        \begin{equation}
                            \left| \bigcup_{g \in G} gHg^{-1} \right| \le 1 + ( |H| -1 ) \left[G: N(H)\right]
                        \end{equation}
                        The first term $1$ on the right-hand side is the identity element of $G$,
                        as the conjugate of the identity element is the identity element itself.
                        The other part of the right-hand side is from counting the conjugate of all the other elements of $H$ in $G$,
                        which two conjugates may give the same element in $G$.

                        \begin{eqnarray}
                            \left| \bigcup_{g \in G} gHg^{-1} \right| &\le& 1 + ( |H| -1 ) \left[G: N(H)\right] \\
                            &=& 1 + ( |H| -1 ) \frac{|G|}{|N(H)|} \\
                            &\le& 1 + ( |H| -1 ) \frac{|G|}{|H|} \\
                            &=& 1 + |G| - \frac{|G|}{|H|} \\
                            &<& |G|
                        \end{eqnarray}

                        The final strict inequality is because $H$ is a proper subgroup of $G$,
                        and indicates that $G$ is not the union of all the conjugates of $H$.
                  \item [(ii)]
                        Let $H =\langle g_{1}, \dots, g_{m} \rangle$.

                        If $H = \left\{1\right\}$, then $G$ has only one conjugacy class, and the conclusion is trivial.

                        If $H = G$, then we prove the conclusion.

                        Otherwise, $H$ is a proper subgroup of $G$.

                        We assert that $\bigcup_{g \in G} gHg^{-1} = G$,
                        as for any $g \in G$, there is a $k$, and $h \in G$,
                        such that $g = h g_{k} h^{-1}$, and $h \in H$.

                        However, this contradicts the conclusion of part i.
              \end{enumerate}
          \end{proof}
\end{enumerate}
\end{document}